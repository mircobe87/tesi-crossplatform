\chapter{Introduzione}

\chapter{Sviluppo Applicazioni Mobili}
	Cosa prendere in considerazione nel momento in cui si decide di sviluppare
	una app; dalle piattaforme da supportare ai tempi e costi che si intendono
	affrontare.
	
	\section{Applicazioni Native}
		Cosa è una app nativa. Cosa comporta lo sviluppo di app native nell'ottica
		di supportare diverse piattaforme. Presentazione dell'approccio "Titanium" che
		permette la realizzazione di app nativa a partire da codice comune javaScript.
		
	\section{Applicazioni Ibride}
		Cosa si intente per app ibrida. Cosa comporta lo sviluppo di questo genere
		di app. Pro e contro.
		
	\section{Applicazioni Web}
		Cosa sono le applicazioni web. Pro e contro.
		
	\section{Conclusioni}
		Visti i pro e contro derivanti dai diversi approcci, considerazioni su
		quale di questi metodi scegliere a seconda dei proprio obiettivi.
		
		
\chapter{Framework Valutati}
	Descrizione tecnologia ibrida e "seminativa" che abbiamo visto.
	Quindi una descrizione generare di tutto un po' tra Cordova/PhoneGap, Titanium,
	Icenium, Sencha Touch, KendoUI Mobile, jQuery Mobile, Rho Mobile e PhoneJS
	con distinzione tra i framework per la UI e per la realizzazione della app vera
	e propria.
	
	\section{Conclusioni}
		Per questo e quello abbiamo scelto di realizzare una app di prova con
		Phonegap e Titanium.


\chapter{L'Applicazione}
	Abbiano concordato un app che fa questo e quello per testare casa si riesce a
	fare e cosa non con i framework scelti.
	\section{Le Tecnologie Utilizzate}
		\subsection{Lato Client}
		\subsection{Lato Server}
	
	\section{Conclusioni}
	Questo è il contenuto del quarto capitolo.
