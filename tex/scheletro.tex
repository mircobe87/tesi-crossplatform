\chapter{Introduzione}

\chapter{Sviluppo Applicazioni Mobili}
	Cosa prendere in considerazione nel momento in cui si decide di sviluppare
	una app; dalle piattaforme da supportare ai tempi e costi che si intendono
	affrontare.
	
	\section{Applicazioni Native}
		Cosa è una app nativa. Cosa comporta lo sviluppo di app native nell'ottica
		di supportare diverse piattaforme. Presentazione dell'approccio "Titanium" che
		permette la realizzazione di app nativa a partire da codice comune javaScript.
		
	\section{Applicazioni Ibride}
		Cosa si intente per app ibrida. Cosa comporta lo sviluppo di questo genere
		di app. Pro e contro.
		
	\section{Applicazioni Web}
		Cosa sono le applicazioni web. Pro e contro.
		
	\section{Conclusioni}
		Visti i pro e contro derivanti dai diversi approcci, considerazioni su
		quale di questi metodi scegliere a seconda dei proprio obiettivi.
		
		
\chapter{Framework Valutati}
	All'azienda in particolare interessava lo studio di framework che permettessero di creare 		    applicazioni distribuibili sui vari store. Ci è stato così chiesto di valutare: 
	Titanium Appcelerator, Phonegap, Icenium, Sencha Touch, KendoUI Mobile, PhoneJS, Rho Mobile
	
	Dobbiamo fare una distinzione, perchè alcuni framework permettono di scrivere applicazioni
	native (sfruttando i linguaggi del web).
	\section{Framework Applicazioni Native}
	\subsection{Titanium Appcelerator}
	Descrivere come funziona titanium e come riesce a creare applicazioni native nonostante 				l'applicazione venga scritta in linguaggio non nativo
	\section{Framework Applicazioni Ibride}
	Facciamo chiarezza sulla distinzione che esiste tra framework che permettono di fare il build 		di un'applicazione e che forniscono le API per l'accesso al dispositivo,  e quelli che servono 		per gestire l'interfaccia grafica
	\subsection{Framework Accesso al dispositivo}
	\subsubsection{Phonegap}
	Descrizione Phonegap
	\subsubsection{Rho Mobile}
	Descrizione Rho Mobile
	\subsubsection{Sencha Touch}
	Descrizione Sencha Touch
	\subsection{Framework per UI}
	\subsubsection{JQuery Mobile}
	Descrizione JQuery Mobile
	\subsubsection{KendoUI Mobile}
	Descrizione KendoUI Mobile
	\subsubsection{PhoneJS}
	Descrizione PhoneJS
	\section{Conclusioni}
		Per questo e quello abbiamo scelto di realizzare una app di prova con
		Phonegap e Titanium.


\chapter{L'Applicazione}
	Abbiamo concordato un app che fa questo e quello per testare cosa si riesce a
	fare e cosa non con i framework scelti (phonegap, titanium).
	\section{Lato Server}
	descrizione funzionalità del server e dei software utilizzati per la sua realizzazione.	
	\section{Lato Client}
	descrizione funzionalità e struttura applicazione. per ogni funzionalità si mostra il confronto 	tra i due
	framework
	\section{Conclusioni}
	Risultato confronto.