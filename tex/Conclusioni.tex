\chapter{Conclusioni}
    Studiare queste tecnologie ci ha permesso di espandere le nostre
    conoscenze nei linguaggi del web, con i quali non avevamo mai avuto
    occasione di entrare in contatto nel nostro percorso di studi, nonchè di
    affacciarci nel mondo dello sviluppo di applicazioni mobili in un periodo in
    cui il suo mercato è in continua espansione.

    Sebbene per iOS
    sia possibile ottenere gratuitamente gli strumenti di sviluppo questi
    possono essere utilizzati soltanto sopra un MAC e il test può avvenire
    soltanto su un simulatore, mentre per poter provare l'applicazione su un
    dispositivo fisico è necessaria la licenza da sviluppatore Apple a
    pagamento. Per ovviare all'utilizzo di un MAC sarebbe stato
    possibile utilizzare il servizio \pg{} Build almeno per l'applicazione
    ibrida, ma
    esso ha ancora bisogno della suddetta licenza. Per l'assenza di quesi
    strumenti l'applicazione è appunto stata sviluppata solo per Android, dove
    gli strumenti di sviluppo sono totalmente gratuiti e dove non serve alcun
    certificato almeno fino a quando non si decide di pubblicare
    l'applicazione su Google Play Store.
    Questo ci ha comunque permesso di approfondire la
    conoscenza del sistema operativo creato da Google Inc., che attualmente è
    tra i più diffusi.

    Come descritto nell'introduzione di questa tesi lo svolgimento del
    tirocinio non ha richiesto la nostra presenza in azienda e quindi il
    lavoro è stato svolto indipendentemente, questo ci ha portato a non avere
    un contatto diretto continuo con il tutore aziendale, con la conseguenza
    che abbiamo dovuto imparare a organizzare il lavoro e a suddividerci i
    compiti. A tal proposito abbiamo colto l'occasione per imparare ad
    utilizzare GIT: uno strumento di VCS (Version Control System) tra i più
    diffusi. Dovendo infatti sviluppare un'applicazione non banale ci era
    sembrato utile l'impiego di uno strumento del genere per gestire il codice,
    e grazie a GITHUB, un servizio di hosting  di repository GIT remoti, abbiamo
    appunto potuto lavorare indipendentemente sulle varie funzionalità
    dell'applicazione senza doverci preoccupare di riunire i vari frammenti
    del codice.

    La realizzazione di questa applicazione con \tisdk{} e \pg{}+\kendomob{}
    ci ha consentito di descrivere solo una parte delle funzionalità che
    questi framework mettono a disposizione. Ad esempio non abbiamo affrontato
    l'integrazione con i social network offerta sia da \tisdk{} che da \pg{}
    oppure in numerosi servizi di backend propri di \tisdk{} come quello per
    monitorare le statistiche di utilizzo dell'applicazione una volta che
    questa viene pubblicata sugli store.

    Inoltre i framework per lo sviluppo di applicazioni \crossplat{} descritti
    in questa tesi sono solo alcuni di quelli disponibili, soprattutto ne
    esistono molti per la creazione di applicazioni web da poter poi usare
    anche per creare applicazioni ibride. Studiarli e capirne i pricipi è
    stato molto utile in quanto, visto il crescente numero di piattaforme
    mobili, quando questa tecnologia sarà veramente pronta e HTML5 abbastanza
    maturo probabilmente diverrà la strada più usata per creare applicazioni
    mobili.
