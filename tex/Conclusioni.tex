\chapter{Conclusioni}
	Studiare queste tecnologie ci ha permesso di espandere le nostre
	conoscenze nei linguaggi del web, con i quali non avevamo mai avuto
	occasione di entrare in contatto nel nostro percorso di studi, nonchè di
	entrare nel mondo dello sviluppo di applicazioni mobili in un periodo in
	cui il suo mercato è in continua espansione.
	
	Purtroppo come detto più volte non è stato possibile provare
	l'applicazione su dispositivi diversi da Android per l'assenza degli
	strumenti opportuni, comunque questo ci ha permesso di approfondire la
	conoscenza del sistema operativo creato da Google Inc., che attualmente è
	il più diffuso.

	Avendo svolto il tirocinio in coppia abbiamo imparato a collaborare,
	soprattutto per quanto riguarda lo sviluppo dell'applicazione, per la
	quale abbiamo fatto uso di git per imparare a bla bla bla. 
	
	Nostre considerazioni conclusive sui risultati ottenuti nello sviluppo 
	dall'applicazione mediante l'uso delle due tecnologie prese in analisi. 
	Considerazioni generali complessive e impressioni sullo sviluppo 
	cross-platform rispetto allo sviluppo nativo.
	per il cross-platform solo studio teorico, no riscontro pratico causa assenza
	strumenti.
	descrizione organizzazione lavoro in team di noi due.
