\chapter{Framework Valutati}
	Il primo passo necessario nella valutazione di questi framework è stato 
	chiarire bene il fine del loro impiego. Durante la fase di ricerca abbiamo 
	notato una gran confusione nell'attribuire a ciascun framework il tipo di
	applicazione che permetteva di realizzare. Spesso il fatto che 
	l'applicazione venisse eseguita all'interno di una componente nativa veniva 
	pubblicizzato come se il framework fosse in grado di creare una completa 
	applicazione nativa (anzichè ibrida); in altri casi 
	veniva confuso il concetto di 
	cross-compiling\hyperref[sec:nativapp]{vedi~\ref{sec:nativapp}} con quello 
	di applicazione ibrida; in altri ancora non veniva mostrata la separazione 
	concettuale che vi è tra framework utili per la sola costruzione della 
	logica e dell'interfaccia grafica dell'applicazione e framework che 
	permettono d'incapsulare il contenuto web in una componente nativa creando 
	così un'applicazione ibrida.
	
	Allo scopo di dare ordine per descrivere al meglio le differenze tra i vari 
	framework abbiamo ritenuto opportuno suddividerli a seconda del tipo di 
	applicazioni che sono in grado di produrre.
	
	Abbiamo così classificato Titanium Appcelerator come framework per la 
	realizzazione di applicazioni native; Phonegap/Cordova, Rho Mobile e Sencha 
	Touch come quelli dediti alla creazione di applicazioni ibride; jQuery 
	Mobile, KendoUI Mobile e PhoneJS utili per implementare complete 
	applicazioni web e per la logica e l'interfaccia grafica di applicazioni 
	ibride.

	\section{Framework per Applicazioni Native}
		
		\subsection{Titanium Appcelerator}
		\label{sec:titanium}
			Titanium è una piattaforma gratis e open source di sviluppo di 
			applicazioni mobili, che permette la creazione di applicazioni native
			cross-platform per iOS, Android, BlackBerry e Tizen usando il 
			linguaggio JavaScript.
			
			
			
			\noindent Titanium è una combinazione di:
			\begin{itemize}
				\item Gli strumenti di Titanium SDK
				\item APIs Mobili
				\item Titanium Studio
				\item Moduli
				\item Servizi cloud di Appcelerator
			\end{itemize}
	
			\subsubsection{Gli strumenti di Titanium SDK}
				Per permettere la costruzione dell'applicazione l'sdk fornisce,
				oltre che a un insieme di script python, degli strumenti di 
				supporto che lavorano insieme alla catena degli strumenti degli
				sdk nativi.
			\subsubsection{APIs Mobili}
			\subsubsection{Titanium Studio}
			\subsubsection{Moduli}
			\subsubsection{Servizi cloud di Appcelerator}
			
	\section{Framework per Applicazioni Web}
	
	\label{sec:frameworkwebapp}
		Come descritto più volte questi framework hanno una duplice funzionalità.
		Ognuno di essi può essere utilizzato indipendentemente per realizzare
		applicazioni web\hyperref[sec:webapp]{vedi~\ref{sec:webapp}}, oppure 
		può essere combinato ai framework descritti nella 
		sezione\hyperref[sec:frameworkhybrid]{~\ref{sec:frameworkhybrid}} per
		realizzare un'applicazione ibrida più soddisfacente. 
		\subsection{JQuery Mobile}
			Descrizione JQuery Mobile
	
		\subsection{KendoUI Mobile}
			Descrizione KendoUI Mobile
	
		\subsection{PhoneJS}
			Descrizione PhoneJS

			
	\section{Framework per Applicazioni Ibride}
	\label{sec:frameworkhybrid}
		\subsection{Phonegap}
			Descrizione Phonegap

		\subsection{Rho Mobile}
			Descrizione Rho Mobile

		\subsection{Sencha Touch}
			Descrizione Sencha Touch
	
	\section{Conclusioni}
		Per questo motivo e quello... abbiamo scelto di realizzare una app di 
		prova con Phonegap e Titanium.
