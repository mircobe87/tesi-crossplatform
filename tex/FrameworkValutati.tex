\chapter{Framework Valutati}
	Il primo passo necessario nella valutazione di questi framework è stato 
	chiarire bene il fine del loro impiego. Durante la fase di ricerca abbiamo 
	notato una gran confusione nell'attribuire a ciascun framework il tipo di
	applicazione che permetteva di realizzare. Spesso il fatto che 
	l'applicazione venisse eseguita all'interno di una componente nativa veniva 
	interpretato come se l'intera applicazione fosse nativa; in altri casi 
	veniva confuso il concetto di 
	cross-compiling\hyperref[sec:nativapp]{vedi~\ref{sec:nativapp}} con quello 
	di applicazione ibrida; a volte ancora non è chiara la separazione 
	concettuale che vi è tra framework utili per la sola costruzione della 
	logica e dell'interfaccia grafica dell'applicazione e framework che 
	permettono d'incapsulare il contenuto web in una componente nativa creando 
	così un'applicazione ibrida.
	
	Allo scopo di dare ordine per descrivere al meglio le differenze tra i vari 
	framework abbiamo ritenuto opportuno suddividerli a seconda del tipo di 
	applicazioni che sono in grado di produrre.
	
	Abbiamo così classificato Titanium Appcellerator come framework per la 
	realizzazione di applicazioni native; Phonegap/Cordova, Rho Mobile e Sencha 
	Touch come quelli dediti alla creazione di applicazioni ibride; jQuery 
	Mobile, KendoUI Mobile e PhoneJS utili per implementare complete 
	applicazioni web.

	\section{Framework per Applicazioni Native}
	
		\subsection{Titanium Appcelerator}
		\label{sec:titanium}
			Descrivere come funziona titanium e come riesce a creare
			applicazioni native nonostante l'applicazione venga scritta in
			linguaggio non nativo
			
	\section{Framework per Applicazioni Ibride}
		Facciamo chiarezza sulla distinzione che esiste tra framework che
		permettono di fare il build di un'applicazione e che forniscono le API
		per l'accesso al dispositivo, e quelli che servono per gestire
		l'interfaccia grafica.
	
		\subsection{Framework di Accesso al dispositivo}
	
			\subsubsection{Phonegap}
				Descrizione Phonegap
	
			\subsubsection{Rho Mobile}
				Descrizione Rho Mobile
	
			\subsubsection{Sencha Touch}
				Descrizione Sencha Touch
	
	\section{Framework per Applicazioni Web}
	\label{sec:frameworkwebapp}
	
		\subsection{JQuery Mobile}
			Descrizione JQuery Mobile
	
		\subsection{KendoUI Mobile}
			Descrizione KendoUI Mobile
	
		\subsection{PhoneJS}
			Descrizione PhoneJS

	\section{Conclusioni}
		Per questo motivo e quello... abbiamo scelto di realizzare una app di 
		prova con Phonegap e Titanium.
