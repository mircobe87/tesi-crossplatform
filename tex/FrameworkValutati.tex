\chapter{Framework Valutati}
	Il primo passo necessario nella valutazione di questi framework è stato 
	chiarire bene il fine del loro impiego. Durante la fase di ricerca abbiamo 
	notato una gran confusione nell'attribuire a ciascun framework il tipo di
	applicazione che permetteva di realizzare. Spesso il fatto che 
	l'applicazione venisse eseguita all'interno di una componente nativa veniva 
	pubblicizzato come se il framework fosse in grado di creare una completa 
	applicazione nativa (anziché ibrida); in altri casi 
	veniva confuso il concetto di 
	cross-compiling (\hyperref[sec:nativapp]{vedi~\ref{sec:nativapp}}) con
	quello di applicazione ibrida; in altri ancora non veniva mostrata la
	separazione concettuale che vi è tra framework utili per la sola costruzione
	della logica e dell'interfaccia grafica dell'applicazione e framework che 
	permettono d'incapsulare il contenuto web in una componente nativa creando 
	così un'applicazione ibrida.
	
	Allo scopo di dare ordine per descrivere al meglio le differenze tra i vari 
	framework abbiamo ritenuto opportuno suddividerli a seconda del tipo di 
	applicazioni che sono in grado di produrre.
	
	Abbiamo così classificato Titanium Appcelerator come framework per la 
	realizzazione di applicazioni native; Phonegap/Cordova, Rho Mobile e Sencha 
	Touch come quelli dediti alla creazione di applicazioni ibride; jQuery 
	Mobile, KendoUI Mobile e PhoneJS utili per implementare complete 
	applicazioni web e per la logica e l'interfaccia grafica di applicazioni 
	ibride.

	\section{Framework per Applicazioni Native}
		
		\subsection{Titanium Appcelerator}
		\label{sec:titanium}
			Titanium è una piattaforma gratis e open source di sviluppo di 
			applicazioni mobili che permette la creazione di applicazioni native
			cross-platform per iOS, Android, BlackBerry e Tizen usando il 
			linguaggio JavaScript.
			
			Titanium è descritto come un framework che adotta l'approccio 
			cross-compiling nel risolvere il problema dello sviluppo per più 
			piattaforme anche se l'uso che si fa di questo appellativo in 
			questo caso non è del tutto corretto. Come spiega lo stesso 
			amministratore delegato di Appcelerator Inc. in un intervento 
			online su \mbox{stackoverflow.com}\footnote{In quell'occasione era 
			stato chiesto proprio come Titanium potesse funzionare riguardo alla 
			creazione del codice nativo. L'intera discussione è consultabile 
			all'indirizzo 
			\url{http://stackoverflow.com/questions/2444001/how-does-appcelerator-titanium-mobile-work}},
			Jeff Haynie, il modo di operare di Titanium è il seguente:
			\begin{quotation}
				Titanium takes your Javascript code, analyzes and preprocesses 
				it and then pre-compiles it into a set of symbols that are 
				resolved based on your applications uses of Titanium APIs. From 
				this symbol hierarchy we can build a symbol dependency matrix 
				that maps to the underlying Titanium library symbols to 
				understand which APIs (and related dependencies, frameworks, 
				etc) specifically your app needs. I'm using the word symbol in a 
				semi-generic way since it's a little different based on the 
				language. In iPhone, the symbol maps to a true C symbol that 
				ultimately maps to a compiled .o file that has been compiled for 
				ARM/i386 architectures. For Java, well, it's more or less a 
				.class file, etc. Once the front end can understand your 
				dependency matrix, we then invoke the SDK compiler (i.e. GCC for 
				iPhone, Java for Android) to then compile your application into 
				the final native binary.
				
				So, a simple way to think about it is that your JS code is 
				compiled almost one to one into the representative symbols in 
				nativeland. There's still an interpreter running in interpreted 
				mode otherwise things like dynamic code wouldn't work. However, 
				its much faster, much more compact and it's about as close to 
				pure native mapping as you can get.
				
				\ldots
			\end{quotation}
			Ciò significa che il sistema fa tutto il possibile per creare codice 
			nativo che rappresenti uno a uno quello che si è descritto in 
			JavaScript ma parte del nostro sorgente dovrà ancora essere 
			interpretato a tempo di esecuzione. Per questo il termine 
			cross-compiling non è del tutto appropriato in quanto con tale 
			aggettivo ci si riferisce ad una tecnica che permette, a partire da 
			un certo sistema, di creare codice eseguibile per un secondo sistema 
			avente caratteristiche e proprietà differenti da quello di 
			partenza\citep{Web:Wiki.cross-compiling}.
			
			In ogni caso possiamo pensare di utilizzare il termine 
			cross-compiling per indicare che Titanium al termine del processo di 
			compilazione dell'applicazione produce un pacchetto che per la 
			maggior parte contiene codice nativo e specifico per una data 
			piattaforma il che è diverso da quello che si ottiene al termine 
			della compilazione di un'applicazione ibrida dove il pacchetto 
			installabile racchiude tutto il codice sorgente web (JavaScript, 
			HTML e CSS) che verrà poi interpretato completamente a tempo di 
			esecuzione da una ``web view'' realizzata 
			\textit{ad~hoc}\footnote{All'interno di questo pacchetto è solo il 
			codice della web view ad essere realizzato in codice nativo e 
			compilato per la tale piattaforma target e null'altro}.
			
			\noindent Detto questo passiamo ad elencare cosa Titanium offre da 
			una visione più ad alto livello\citep[Cap.2 - Titanium 
			Mobile Overview]{Book:Ti}:
			\begin{itemize}
				\item Strumenti dello SDK Titanium
				\item APIs Mobile
				\item Titanium Studio
				\item Moduli
				\item Servizi cloud di Appcelerator
			\end{itemize}
	
			\subsubsection{Strumenti dello SDK Titanium}
				Come strumenti per compilare l'applicazione in un pacchetto 
				installabile contenente codice nativo, Titanium SDK utilizza un 
				insieme di script Python e altri strumenti di supporto che 
				lavoreranno insieme a quelli forniti dagli SDK per lo sviluppo 
				nativo\footnote{Da qui la necessità di lavorare in un ambiente 
				configurato con gli SDK nativi delle piattaforme per le quali si 
				desidera realizzare l'applicazione}. Tutto questo è trasparente 
				agli occhi di uno sviluppatore che può così concentrarsi solo 
				nella realizzazione della propria applicazione.
				
			\subsubsection{APIs Mobili}
				Titanium fornisce un ricco insieme di API JavaScript che danno 
				accesso a centinaia tra componenti native per l'interfaccia 
				utente e componenti non visuali. Queste API sono suddivise in 
				vari insiemi come Titanium.UI (per quanto riguarda l'interfaccia 
				utente) o Titanium.Network (per quanto riguarda il networking).
				 
			\subsubsection{Titanium Studio}
				Appcelerator mette anche a disposizione un IDE gratuito per lo 
				sviluppo tramite Titanium SDK. Titanium Studio, appunto, è un 
				ambiente di sviluppo integrato derivato da Eclipse, uno degli 
				IDE open-source più utilizzati. Con Titanium Studio è possibile 
				scrivere, fare testing e debugging delle proprie applicazioni; 
				inoltre sono presenti anche vari templates e applicazioni di 
				esempio per rendere più semplice incominciare a creare 
				applicazioni con Titanium SDK. Titanium Studio permette anche di 
				gestire gli aggiornamenti dello SDK.
				
				L'uso di questo strumento non è obbligatorio, Titanium SDK 
				ha una propria interfaccia a linea di comando che permette di 
				gestire ogni aspetto dello sviluppo: inizializzare un 
				nuovo progetto, compilarlo, eseguirlo e permette anche di 
				installare nuovi aggiornamenti dello SDK nonché di configurarlo.
				Ovviamente se si rinuncia a Titanium Studio non è garantito che 
				si riescano a trovare i medesimi tool per il debugging in altri 
				IDE che si possono trovare in quello fornito da Appcelerator.
				
			\subsubsection{Moduli}
				Titanium è composto da una serie di moduli che estendono alcune 
				funzioni principali delle API. Se si controlla la documentazioni 
				relativa di può trovare una lista di moduli base che estendono 
				il nucleo del sistema e, inoltre, Appcelerator pubblica 
				campioni di moduli gratuiti sul proprio repository git su 
				\mbox{github.com}\footnote{Il repository ufficiale di tutti i 
				moduli pubblicati è	raggiungibile all'indirizzo 
				\url{https://github.com/appcelerator/titanium_modules}}. 
				Ovviamente ogni sviluppatore è libero di creare e distribuire 
				gratuitamente o vendere i proprio moduli attraverso il 
				Marketplace\footnote{\url{https://marketplace.appcelerator.com/}} 
				di Appcelerator. Titanium SDK supporta l'impiego dei moduli 
				anche nello sviluppo delle applicazioni web, in questo caso la 
				loro realizzazione consiste nello scrivere un puro modulo 
				JavaScript e non qualcosa scritto in Java o Objective-C.
				
			\subsubsection{Servizi cloud di Appcelerator}
				Appcelerator fornisce svariati servizi ACS (Appcelerator Cloud 
				Services) che sono inclusi nella lista delle varie componenti 
				fruibili attraverso le API. Alcune delle funzionalità offerte da 
				questi servizi sono:
				\begin{itemize}
					\item Invio di notifiche push
					\item Gestione dell'utente
					\item Salvataggio e manipolazione di foto
					\item Integrazioni con i social network
					\item Memorizzazione di file
					\item Chat
					\item Memorizzazione di dati in formato chiave-valore
				\end{itemize}
				Per poter usufruire dei servizi ACS nella propria applicazione, 
				otre che ad utilizzare le relative API fornite è necessario 
				registrare la propria app online sul sistema di gestione dei 
				servizi ACS di Appcelerator.
		
	\section{Framework per Applicazioni Web}
	
	\label{sec:frameworkwebapp}
		Come descritto più volte questi framework hanno una duplice funzionalità.
		Ognuno di essi può essere utilizzato indipendentemente per realizzare
		applicazioni web (\hyperref[sec:webapp]{vedi~\ref{sec:webapp}}), oppure 
		può essere combinato ai framework descritti nella 
		sezione\hyperref[sec:frameworkhybrid]{~\ref{sec:frameworkhybrid}} per
		realizzare un'applicazione ibrida più soddisfacente. 
		\subsection{JQuery Mobile}
			Descrizione JQuery Mobile
	
		\subsection{KendoUI Mobile}
			Descrizione KendoUI Mobile
	
		\subsection{PhoneJS}
			Descrizione PhoneJS

			
	\section{Framework per Applicazioni Ibride}
	\label{sec:frameworkhybrid}
		\subsection{Phonegap}
			Descrizione Phonegap

		\subsection{Rho Mobile}
			Descrizione Rho Mobile

		\subsection{Sencha Touch}
			Descrizione Sencha Touch
	
	\section{Conclusioni}
		Per questo motivo e quello... abbiamo scelto di realizzare una app di 
		prova con Phonegap e Titanium.
