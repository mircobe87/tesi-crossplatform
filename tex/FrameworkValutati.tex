\chapter{Framework Valutati}
	All'azienda in particolare interessava lo studio di framework che
	permettessero di creare applicazioni distribuibili sui vari store.
	Ci è stato così chiesto di valutare: Titanium Appcelerator, Phonegap,
	Icenium, Sencha Touch, KendoUI Mobile, PhoneJS, Rho Mobile, jQuery Mobile.
	
	Dobbiamo fare una distinzione, perchè alcuni framework permettono di
	scrivere applicazioni native (sfruttando i linguaggi del web).
	
	\section{Framework Applicazioni Native}
	
		\subsection{Titanium Appcelerator}
			Descrivere come funziona titanium e come riesce a creare
			applicazioni native nonostante l'applicazione venga scritta in
			linguaggio non nativo
			
	\section{Framework Applicazioni Ibride}
		Facciamo chiarezza sulla distinzione che esiste tra framework che
		permettono di fare il build di un'applicazione e che forniscono le API
		per l'accesso al dispositivo, e quelli che servono per gestire
		l'interfaccia grafica.
	
		\subsection{Framework Accesso al dispositivo}
	
			\subsubsection{Phonegap}
				Descrizione Phonegap
	
			\subsubsection{Rho Mobile}
				Descrizione Rho Mobile
	
			\subsubsection{Sencha Touch}
				Descrizione Sencha Touch
	
			\subsection{Framework per UI}
	
				\subsubsection{JQuery Mobile}
					Descrizione JQuery Mobile
	
				\subsubsection{KendoUI Mobile}
					Descrizione KendoUI Mobile
	
				\subsubsection{PhoneJS}
					Descrizione PhoneJS

	\section{Conclusioni}
		Per questo motivo e quello... abbiamo scelto di realizzare una app di 
		prova con Phonegap e Titanium.
