\setcounter{page}{1}
\pagenumbering{arabic}

\chapter{Introduzione}
    L'azienda Intecs S.p.A di Pisa, che sviluppa anche applicazioni mobili per la
    piattaforma Android, talvolta riceve richieste di supportare altre piattaforme,
    principalmente iOS. Da qui nasce l'interesse da parte dell'azienda nel
    cercare un metodo di sviluppo \crossplat{} che eviti di dover acquisire
    conoscenze specifiche nello sviluppo di applicazioni native per le altre
    piattaforme. L'azienda aveva già effettuato alcune prove con le cosiddette
    applicazioni ibride ma con risultati poco soddisfacenti almeno dal punto di
    vista delle performance.

    La richiesta dell'azienda è stata quindi quella di approfondire lo studio sui
    framework per la realizzazione di applicazioni ibride e/o di trovare alternative
    in grado di sostituirsi alla scrittura diretta di applicazioni in codice
    nativo.

    Il tirocinio nato per essere svolto da una persona è stato poi esteso a due
    tirocinanti, rendendo possibile valutare un maggior numero di framework e
    realizzare un'applicazione più complessa.

    Il lavoro si è svolto in tre fasi. La prima fase è stata necessaria per
    assimilare i concetti base sulle tecnologie che sono prerequisiti per lo
    svolgimento del tirocinio, in particolare è stata presa manualità con i
    linguaggi del web (HTML5, CSS3, JavaScript). La seconda fase, puramente di
    ``ricerca'', ha avuto lo scopo di raccogliere informazioni sulle diverse
    tecniche di programmazione per applicazioni mobili. In particolare sono
    stati analizzati i framework: \tisdk{},
    PhoneGap/Cordova e RhoMobile, come richiesto dall'azienda, e i framework:
    jQuery Mobile, PhoneJS, KendoUI Mobile e SenchaTouch che abbiamo ritenuto
    utili a rendere più completo il tirocinio.
    Presa conoscenza delle funzionalità dei suddetti framework abbiamo fatto un
    resoconto all'azienda e, sulla base di queste informazioni, il tutore
    aziendale ci ha chiesto di realizzare un'applicazione con \pg{} e con
    \tisdk{} in modo da poter fare un confronto pratico. Come verrà spiegato
    nel seguito di questo documento \pg{} necessita di essere accompagnato da
    un altro framework per realizzare l'interfaccia grafica
    dell'applicazione, la scelta di quest'ultimo è stata lasciata a noi che
    a tale scopo abbiamo utilizzato \kendomob{}.
    La terza fase si è quindi concentrata sulla realizzazione di questa
    applicazione nelle sue due differenti versioni. La richiesta di aggiungere
    funzionalità all'app durante la sua implementazione, ci ha permesso anche di
    valutare quale dei due framework permette la scrittura di codice più
    manutenibile.

    Per lo sviluppo di questo tirocinio non è stata necessaria la presenza in
    azienda, e la collaborazione col tutore aziendale è avvenuta principalmente
    attraverso posta elettronica.

    La tesi, dunque, avrà una parte iniziale nella quale vengono mostrati i
    diversi tipi di approccio allo sviluppo di applicazioni mobili, seguita dalla
    descrizione dei framework presi in esame. Verrà poi descritta l'applicazione
    realizzata evidenziando quale dei due framework si è comportato meglio nel
    rendere possibile l'implementazione delle diverse funzionalità che il
    tutore aziendale ci ha chiesto di realizzare. La tesi concluderà con un capitolo
    che raccoglie il risultato del lavoro svolto.

