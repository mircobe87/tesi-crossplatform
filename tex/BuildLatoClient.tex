\chapter{Compilazione Del Lato Client}

In questa sezione vediamo molto rapidamente come creare i pacchetti installabili
per Android del lato client dell'applicazione, prima la versione realizzata con
\pg{} e poi quella realizzata con \tisdk{}.

\section{Android SDK}
    Come dicevamo, compilando l'applicazione per Android abbiamo la necessità di
    dover installare nel sistema l'SDK nativo di Android.
    
    Il pacchetto software è disponibile in rete all'url
    \url{http://dl.google.com/android/android-sdk_r22.6.2-linux.tgz} e, una
    volta scaricato, dobbiamo estrarlo all'interno di una directory a piacere;
    il comando seguente mostra come estrarre il pacchetto scaricato dalla 
    cartella \texttt{Scaricati} nella cartella home:
    \begin{lstlisting}[language=plane]
 cd ~
 tar -xvf ./Scaricati/android-sdk_r22.6.2-linux.tgz
    \end{lstlisting}
    Terminata l'estrazione, troveremo la nuova cartella
    \texttt{android-sdk-linux}. Adesso è necessario aggiungere alla 
    variabile d'ambiente \texttt{PATH} i percorsi alle cartelle contenenti i 
    tool di compilazione ed installazione di Android SDK; questo è possibile 
    farlo con il seguente comando:
    \begin{lstlisting}[language=plane]
 echo 'export PATH=$PATH:~/android-sdk-linux/tools:~/android-sdk-linux/platform-tools"' >> ~/.bashrc
 source ~/.bashrc
    \end{lstlisting}
    A questo punto dobbiamo installare la versione delle API di Android che 
    vogliamo e per avviare il tool è sufficiente usare il seguente comando:
    \begin{lstlisting}[language=plane]
    android &
    \end{lstlisting}
     

\section{\pg{}}
    Vediamo ora come istallare e configurare \pg{}.
    
    \subsection{Installare \pg{}}
        Prima di tutto è necessario aver installato Node.js come descritto in 
        \ref{sec:nodejs}.
        
        Installare \pg{} è molto semplice, basta installare il relativo modulo 
        Node.js con il seguente comando:
        \begin{lstlisting}[language=plane]
 sudo npm install -g phonegap
        \end{lstlisting}
        Al termine dell'installazione il comando \texttt{phonegap} sarà 
        disponibile.
        
    \subsection{Compilare L'Applicazione}
        Ora che \pg{} è installato dobbiamo scaricare il codice sorgente 
        dell'applicazione, installare i plugin di \pg{} necessari e a questo 
        punto si può procedere con la compilazione ed eventuale installazione 
        sul dispositivo.
        
        \subsubsection{Scaricare il codice sorgente}
            Il codice sorgente dell'applicazione \pg{} è disponibile al 
            repository git 
            \url{https://github.com/mircobe87/degradoAmbientale}; se si ha 
            \texttt{git} installato è sufficiente dare il comando seguente per 
            ottenere i sorgenti necessari:
            \begin{lstlisting}[language=plane]
 git clone https://github.com/mircobe87/degradoAmbientale.git
            \end{lstlisting}
            Al termine del download tutto il progetto sarà presente nella 
            cartella \texttt{degradoAmbientale}.
            
        \subsubsection{Installare i plugin necessari}
            Prima di procedere con la compilazione è necessario installare i 
            plugin utilizzati dall'applicazione. Per agevolare 
            l'installazione, all'interno della cartella 
            \texttt{degradoAmbientale}, è presente lo script Bash 
            \texttt{install-plugins.sh} da lanciare con il seguente comando:
            \begin{lstlisting}[language=plane]
 cd degradoAmbientale
 ./install-plugins.sh
            \end{lstlisting}
            Questo provvederà ad installare tutti i plugin ma se si vuole 
            eseguire l'operazione manualmente allora è necessario prima creare 
            l'apposita cartella \texttt{degradoAmbientale/plugins} e 
            successivamente installare uno alla volta i plugin:
            \begin{itemize}
                \item org.apache.cordova.device@0.2.7
                \item org.apache.cordova.network-information@0.2.6
                \item org.apache.cordova.geolocation@0.3.5
                \item org.apache.cordova.camera@0.2.6
                \item org.apache.cordova.file@0.2.5
                \item org.apache.cordova.file-transfer@0.4.0
                \item org.apache.cordova.splashscreen@0.2.6
                \item org.apache.cordova.console@0.2.6
            \end{itemize}
            Per installare un plugin dal nome \texttt{"plugin-name"} è 
            sufficiente usare il comando
            \begin{lstlisting}[language=plane]
 phonegap local plugin add "plugin-name"
            \end{lstlisting}
            
            È importante sottolineare che tutti i comandi
            \texttt{phonegap\ldots} vanno dati all'interno della cartella 
            ``root'' del progetto.
            
            Un altro accorgimento da notare è che per poter installare 
            l'applicazione sul dispositivo è necessario che l'opzione di 
            sviluppo \texttt{Debug USB} si attiva. 
            
        \subsubsection{Compilare e installare}
            Adesso siamo pronti per creare l'applicazione. \pg{} dà la 
            possibilità di compilare l'applicazione e successivamente 
            installarla oppure è possibile con un unico comando eseguire 
            entrambe le operazioni. Per compilare l'applicazione per Android 
            bisogna dare il comando:
            \begin{lstlisting}[language=plane]
 phonegap build android
            \end{lstlisting}
            Al termine dell'operazione, una volta connesso il dispositivo alla 
            macchina, è possibile installare l'applicazione con il comando:
            \begin{lstlisting}[language=plane]
 phonegap install android
            \end{lstlisting}
            
            Se si vogliono eseguire insieme queste due operazioni si può 
            utilizzare il comando:
            \begin{lstlisting}[language=plane]
 phonegap run android
            \end{lstlisting}

\section{\tisdk{}}

    \subsection{Installare e Configurare \tisdk{}}
    Anch'esso richiede l'installazione di Node.js descritta più volte nei
    paragrafi precedenti. I comandi per installare l'ultima versione stabile
    di \tisdk{} sono i seguenti:
    \begin{lstlisting}[language=plane]
 sudo npm install -g titanium
 titanium sdk install --default
    \end{lstlisting}
        \subsubsection{Configurare Titanium CLI}
        Prima di poter utilizzare Titanium CLI è necessario configurarlo con
        il comando:
    \begin{lstlisting}[language=plane]
 titanium setup
    \end{lstlisting}
    Esso mostrerà alcune domande come l'indirizzo mail, il percorso nel quale
    è memorizzato l'Android SDK, e altre cose.

    Inoltre è possibile (anche se non necessario) installare l'IDE Titanium
    Studio seguendo le istruzioni dal link \url{http://docs.appcelerator.com/titanium/latest/#!/guide/Quick_Start}
    \subsection{Compilare L'Applicazione}
    In seguito verrà mostrato come compilare ed installare l'applicazione
    utilizzando la Titanium CLI.
        \subsubsection{Scaricare il codice sorgente}
        Il codice sorgente dell'applicazione \tisdk{} è disponibile al 
            repository git 
            \url{https://github.com/pratesim/drawerTitanium.git}; se si ha 
            \texttt{git} installato è sufficiente dare il comando seguente per 
            ottenere i sorgenti necessari:
            \begin{lstlisting}[language=plane]
 git clone https://github.com/pratesim/drawerTitanium.git
            \end{lstlisting}
            Al termine del download tutto il progetto sarà presente nella 
            cartella \texttt{drawerTitanium}.
        \subsubsection{Compilare e installare}
            Con \tisdk{} è possibile compilare e installare direttamente
            l'applicazione su un dispositivo fisico o su un emulatore, oppure
            è possibile eseguire solo la compilazione.
            Per compilare ed installare l'applicazione per Android 
            bisogna dare il comando:
        \begin{lstlisting}[language=plane]
 titanium build --platform android
        \end{lstlisting}
        In questo modo se vi è un dispositivo Android collegato al PC
        l'applicazione viene installata su tale dispositivo, mentre se questo
        non è presente allora viene avviato un emulatore.

        Se vogliamo evitare l'installazione e limitarci alla sola compilazione
        bisogna eseguire il comando:
        \begin{lstlisting}[language=plane]
 titanium build --build-only --platform android
        \end{lstlisting}
