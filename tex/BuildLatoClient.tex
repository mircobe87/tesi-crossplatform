\chapter{Compilazione Del Lato Client}

In questa sezione vediamo molto rapidamente come creare i pacchetti installabili
per Android del lato client dell'applicazione, prima la versione realizzata con
\pg{} e poi quella realizzata con \tisdk{}.

\section{Android SDK}
    Come dicevamo, compilando l'applicazione per Android abbiamo la necessità di
    dover installare nel sistema l'SDK nativo di Android.
    
    Il pacchetto software è disponibile in rete all'url
    \url{http://dl.google.com/android/android-sdk_r22.6.2-linux.tgz} e, una
    volta scaricato, dobbiamo estrarlo all'interno di una directory a piacere;
    il comando seguente mostra come estrarre il pacchetto scaricato dalla 
    cartella \texttt{Scaricati} nella cartella home:
    \begin{lstlisting}[language=plane]
    cd ~
    tar -xvf ./Scaricati/android-sdk_r22.6.2-linux.tgz
    \end{lstlisting}
    Terminata l'estrazione, troveremo la nuova cartella
    \texttt{android-sdk-linux}. Adesso è necessario aggiungere alla 
    variabile d'ambiente \texttt{PATH} i percorsi alle cartelle contenenti i 
    tool di compilazione ed installazione di Android SDK; questo è possibile 
    farlo con il seguente comando:
    \begin{lstlisting}[language=plane]
    echo 'export PATH=$PATH:~/android-sdk-linux/tools:~/android-sdk-linux/platform-tools"' >> ~/.bashrc
    source ~/.bashrc
    \end{lstlisting}
    A questo punto dobbiamo installare la versione delle API di Android che 
    vogliamo e per avviare il tool è sufficiente usare il seguente comando:
    \begin{lstlisting}[language=plane]
    android &
    \end{lstlisting}
     

\section{\pg{}}
    Vediamo ora come istallare e configurare \pg{}.
    
    \subsection{Installare \pg{}}
        Prima di tutto è necessario aver installato Node.js come descritto in 
        \ref{sec:nodejs}.
        
        Installare \pg{} è molto semplice, basta installare il relativo modulo 
        Node.js con il seguente comando:
        \begin{lstlisting}[language=plane]
    sudo npm install -g phonegap
        \end{lstlisting}
        Al termine dell'installazione il comando \texttt{phonegap} sarà 
        disponibile.
        
    \subsection{Compilare L'Applicazione}
        Ora che \pg{} è installato dobbiamo scaricare il codice sorgente 
        dell'applicazione, installare i plugin di \pg{} necessari e a questo 
        punto si può procedere con la compilazione ed eventuale installazione 
        sul dispositivo.
        
        \subsubsection{Scaricare il codice sorgente}
            Il codice sorgente dell'applicazione \pg{} è disponibile al 
            repository git 
            \url{https://github.com/mircobe87/degradoAmbientale}; se si ha 
            \texttt{git} installato è sufficente dare il comando seguente per 
            ottenere i sorgenti necessari:
            \begin{lstlisting}[language=plane]
    git clone https://github.com/mircobe87/degradoAmbientale.git
            \end{lstlisting}
            Al termine del dowload tutto il progetto sarà presente nella 
            cartella \texttt{degradoAmbientale}.
            
        \subsubsection{Installare i plugin necessari}
            Prima di procedere con la compilazione è necessario installare i 
            plugin utilizzati dall'applicazione. Per agevolare 
            l'installazione, all'interno della cartella 
            \texttt{degradoAmbientale}, è presente lo script Bash 
            \texttt{install-plugins.sh} da lanciare con il seguente comando:
            \begin{lstlisting}[language=plane]
    cd degradoAmbientale
    ./install-plugins.sh
            \end{lstlisting}
            
        \subsubsection{Compilare e installare}


\section{\tisdk{}}

    \subsection{Installare e Configurare \tisdk{}}
    Anch'esso richiede l'installazione di Node.js descritta più volte nei
    paragrafi precedenti. I comandi per installare l'ultima versione stabile
    di \tisdk{} sono i seguenti:
    \begin{lstlisting}[language=plane]
sudo npm install -g titanium
titanium sdk install --default
    \end{lstlisting}
        \subsubsection{Configurare Titanium CLI}
        Prima di poter utilizzare Titanium CLI è necessario configurarlo con
        il comando:
        begin{lstlisting}[language=plane]
titanium setup
    \end{lstlisting}
    Esso mostrerà alcune domande come l'indirizzo mail, il percorso nel quale
    è memorizzato l'Android SDK, e altre cose.
    
    \subsection{Compilare L'Applicazione}
        \subsubsection{Scaricare il codice sorgente}
        \subsubsection{Compilare e installare}
