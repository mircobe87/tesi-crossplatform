\chapter{Compilazione Del Lato Client}

In questa sezione vediamo molto rapidamente come creare i pacchetti installabili
per Android del lato client dell'applicazione, prima la versione realizzata con
\pg{} e poi quella realizzata con \tisdk{}.

\section{Android SDK}
    Come dicevamo, compilando l'applicazione per Android abbiamo la necessità di
    dover installare nel sistema l'SDK nativo di Android.
    
    Il pacchetto software è disponibili in rete all'url
    \url{http://dl.google.com/android/android-sdk_r22.6.2-linux.tgz} e, una
    volta scaricato, dobbiamo estrarlo all'interno di una directory a piacere;
    il comando seguente mostra come estrarre il pacchetto nella cartella home:
    \begin{lstlisting}[language=plane]
    cd ~
    tar -xvf ./Scaricati/android-sdk_r22.6.2-linux.tgz
    \end{lstlisting}

\section{\pg{}}

    \subsection{Installare \pg{}}

    \subsection{Compilare L'Applicazione}
        \subsubsection{Scaricare il codice sorgente}
        \subsubsection{Installare i plugin necessari}
        \subsubsection{Compilare e installare}


\section{\tisdk{}}

    \subsection{Installare e Configurare \tisdk{}}
        \subsubsection{Installare Titanium CLI}
        \subsubsection{Installare \tisdk{}}
        \subsubsection{Configurare \tisdk{}}
        
    \subsection{Compilare L'Applicazione}
        \subsubsection{Scaricare il codice sorgente}
        \subsubsection{Compilare e installare}
