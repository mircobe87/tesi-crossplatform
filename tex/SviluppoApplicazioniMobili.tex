\chapter{Sviluppo Applicazioni Mobili}
	Vista la grande diffusione di dispositivi mobili, come smartphone e tablet, 
	in grado di eseguire applicazioni scritte da terze parti, un'azienda 
	potrebbe essere intenzionata a fornire i propri servizi anche attraverso 
	queste applicazioni.
	
	Diversamente dalle applicazioni per desktop, dove per molto tempo la 
	piattaforma di riferimento è stata Windows, il panorama delle piattaforme 
	mobili è molto frammentato. I principali sistemi operativi mobili sono: iOS, 
	Android e Windows Phone, seguiti dai meno diffusi Blackberry e Symbian, 
	inoltre ne stanno nascendo continuamente di nuovi.
	
	Quando si decide di sviluppare un'applicazione mobile, è quindi necessario 
	scegliere se sviluppare per una o più di queste piattaforme. Lo sviluppo su 
	una sola piattaforma (sviluppo \singleplat), è appropriato se si ha 
	intenzione di sviluppare applicazioni che avranno un uso interno dove i 
	dispositivi sono controllati o se per un strategia di mercato si è deciso di 
	concentrarci sullo sviluppo per un unico produttore che adotta un'unica 
	piattaforma. Anche se sviluppare per una singola piattaforma semplifica di 
	molto il lavoro, questo pone dei rischi in quanto il successo 
	dell'applicazione è legato al mantenimento del grado di preferenza degli 
	utenti della piattaforma supportata. L'alternativa è quella di sviluppare 
	l'applicazione supportando diverse piattaforme (sviluppo \crossplat), 
	col vantaggio di raggiungere un bacino di utenza più ampio.
	
	Ci sono molti approcci differenti per lo sviluppo \crossplat, ognuno con 
	differenti vantaggi, ma la caratteristica che tutti hanno in comune è la 
	possibilità di sviluppare la stessa (o molto simile) applicazione per 
	diverse piattaforme mobili.
	 
	Principalmente è possibile sviluppare tre tipi di applicazioni, ognuno dei 
	quali può essere usato sia per lo sviluppo \singleplat, sia per quello 
	\crossplat: Native, Web e Ibride. Nei successivi paragrafi 
	verranno spiegati nel dettaglio, e per ognuno saranno affrontati i pro e i 
	contro, con particolare attenzione per lo sviluppo \crossplat.
	
	\section{Applicazioni Native}
	\label{sec:nativapp}
		Lo sviluppo nativo di applicazioni si basa sull'utilizzo di SDK e di 
		linguaggi di programmazione specifici per ogni piattaforma. Per esempio, 
		le applicazioni native per iOS usano le API di Apple, Objective C e 
		UIKit; su Android si utilizzano le API di Google, Java e la specifica 
		sintassi XML per la UI; su Windows Phone si utilizzano le API di 
		Microsoft, .NET e XAML. Ogni SDK fornisce gli strumenti necessari per 
		compilare il codice sorgente in codice eseguibile sulla propria 
		piattaforma, per creare pacchetti installabili\footnote{Su Android le 
		applicazioni vendono distribuite in pacchetti .apk, su iOS in .ipa e su 
		Windows Phone 8 in .xap} e diversi strumenti utili per il debugging.

		Sviluppare applicazioni native permette di creare applicazioni molto 
		performanti, tale caratteristica rende questa tecnica la più opportuna 
		per lo sviluppo di giochi con una ricca grafica. Inoltre grazie alle 
		API, fornite direttamente dal produttore, si ha un completo accesso al 
		dispositivo, con la possibilità quindi di sfruttare i vari sensori 
		(come fotocamera, GPS, accelerometro). Infine è possibile distribuire 
		l'applicazione sfruttando i vari store\footnote{Ogni piattaforma 
		fornisce il proprio canale di distribuzione per le applicazioni: 
		App Store per iOS, Google Play Store per Android, Windows Store per 
		Windows Phone} delle piattaforme, questo permette di avere maggiore 
		pubblicità per l'applicazione grazie all'enorme ecosistema che si è 
		venuto a creare intorno ad essi.

		Di contro questa tecnica non è molto adatta allo sviluppo di 
		applicazioni \crossplat. Sviluppare la stessa applicazione per 
		diverse piattaforme richiede che diversi team di sviluppatori scrivano 
		la stessa applicazione più volte in linguaggi diversi e in ambienti di 
		sviluppo differenti. Il problema inoltre cresce con il numero di 
		piattaforme che si intende supportare. Un'azienda, per ogni piattaforma, 
		deve spendere lo stesso tempo, sforzo e denaro per produrre e testare 
		l'applicazione. E lo stesso vale anche per il mantenimento della app. 
		Un' altro importante fattore in ogni progetto di sviluppo di 
		applicazioni native \crossplat è la disponibilità di programmatori 
		con le giuste conoscenze. Lo sviluppo di applicazioni native richiede 
		l'avere a disposizione un gruppo di sviluppatori che conosca tutti i 
		vari linguaggi necessarri per lavorare su ogni piattaforma (requisito 
		non	molto facile da soddisfare) oppure si dovrà contare un diversi team 
		di programmatori ognuno specializzato su una specifica piattaforma.
		
		Esistono strumenti di cross-compiling che possono essere una soluzione 
		parziale del problema consentendo di scrivere l'applicazione in un solo 
		linguaggio (per esempio javaScript) e poi compilarla in applicazione 
		nativa. Questa soluzione aiuta a ridurre il numero di differenti 
		linguaggi di programmazione che gli sviluppatori devono saper maneggiare 
		ma comunque non elimina la necessità di dover compilare differenti 
		applicazioni per ogni piattaforma. Cross-compiling inoltre introduce un 
		nuovo livello di astrazione tra il codice della app e il runtime; di 
		questo bisogna tenere conto quando si vanno a fare operazioni di 
		ottimizzazione, testing e di debugging.

	\section{Applicazioni Web}
	\label{sec:webapp}
		Le Applicazioni Web (chiamate anche Browser App o Web App) non sono 
		altro che siti web ottimizzati per i dispositivi mobili. Il loro uso è 
		reso possibile dai moderni browser mobili come: Mobile Safari (sui 
		dispositivi iOS), Google Chrome per Android (sui dispositivi Android 4+) 
		e Internet Explorer 10 (sui dispositivi Windows Phone 8).
		
		Per realizzare questo tipo di applicazione viene sfruttata la potenza di 
		HTML5, CSS3 e JavaScript. HTML5 è la quinta versione del linguaggio 
		HTML, il quale fornisce gli elementi di base che costituiscono le pagine 
		web. HTML5 supporta i contenuti multimediali, ed è stato creato con 
		l'obiettivo di poter essere eseguito da qualsiasi browser 
		mobile\citep{White:Native-vs-Html}. Una delle tante novità di HTML5 è la 
		cosiddetta ``application cache'': attraverso questo strumento è 
		possibile dire al browser quali file salvare (non nella cache 
		``normale'' ma in un cache apposita, definita per l'appunto 
		``application cache''). Il funzionamento dell'application cache è 
		piuttosto semplice: sostanzialmente, si crea un file all'interno del 
		quale si specifica la lista di file che il browser deve salvare nella 
		sua ``memoria'' e che deve mostrare anche quando si è offline. Il 
		vantaggio principale della ``application cache'', rispetto a quella 
		tradizionale, è il controllo: mentre nella cache ``normale'' è il 
		browser a decidere (più o meno) quali sono i file da salvare e tenere in 
		cache, con lo strumento offerto da HTML5 possiamo dire noi con 
		precisione quali risorse tenere in memoria, possiamo infatti decidere di 
		memorizzare pagine sulle quali non si è ancora navigato, fino 
		addirittura a memorizzare interi script. In questo modo le applicazioni 
		web sono in grado di essere eseguite anche quando il dispositivo mobile 
		è offline.

		Scrivere un'applicazione web di successo, non è solo una questione di
		rispettare le specifiche, ma dipende anche dall'implementazione del 
		browser del dispositivo. È da vedere quanto sia nell'interesse di 
		aziende come Apple e Google tenere i loro browser aggiornati 
		continuamente con gli standard, infatti ci sarebbe poi il rischio che 
		HTML5 diventi uno standard per le applicazioni e che quindi ci rimettano 
		gli store\citep{White:Native-vs-Html}. 

		Sfruttando opportunamente CSS3, HTML5 e JavaScript, gli sviluppatori 
		possono ottimizzare la loro applicazione presentandola con un aspetto 
		diverso a seconda del sistema operativo mobile che intendono supportare, 
		in modo da rendere l'esperienza utente il più possibile simile a quella 
		attesa, oppure possono progettare un'unica interfaccia grafica che viene 
		utilizzata su ogni piattaforma. A tale scopo sono nati numerosi 
		framework (\hyperref[sec:frameworkwebapp]{vedi~\ref{sec:frameworkwebapp}}).

		HTML5 promette che scrivendo un unico codice per l'applicazione, 
		questa sia direttamente funzionante su ogni piattaforma, questa e le 
		altre caratteristiche rendono la creazione di web app il modo più veloce 
		per realizzare applicazioni mobili  \crossplat. Essendo eseguite 
		direttamente nel browser non c'è la necessità di creare il pacchetto 
		nativo da installare su ogni piattaforma, liberando così gli 
		sviluppatori dal vincolo di lavorare in differenti ambienti di sviluppo. 
		Le applicazioni web hanno il vantaggio di poter fruire di aggiornamenti 
		direttamente attraverso il browser, è infatti sufficiente che lo 
		sviluppatore aggiorni il codice della pagina web online. Non sono quindi
		necessari passaggi extra per la distribuzione dell'aggiornamento. 
		Rispetto alle applicazioni native c'è il vantaggio della maggior 
		disponibilità di programmatori con conoscenze dei linguaggi CSS3, HTML5 
		e JavaScript, rispetto a quelli con conoscenze specifiche per i 
		linguaggi Objective-C, Java e .NET.

		Il più grande svantaggio di questo genere di applicazioni è che non 
		possono accedere ad una vasta gamma di sensori e API dei vari 
		dispositivi. La promessa di HTML5 di scrivere una volta ed eseguire 
		ovunque non è totalmente rispettata, infatti, essendoci molti browser 
		diversi e molte versioni degli stessi browser (chrome è stato aggiornato 
		21 volte in due anni, con conseguente modifica dell'interpretazione di 
		web app\citep{White:Native-vs-Html}) i vari dispositivi accedono in modo 
		diverso alle applicazioni web. La conseguenza è che gli sviluppatori 
		devono creare versioni diverse per browser diversi, con la conseguente 
		perdita di parte dei benefici argomentati per lo sviluppo di web app. Un 
		altro grande svantaggio si ha nelle performance che sono sicuramente 
		inferiori a quelle delle applicazioni native, dato che le prestazioni 
		delle applicazioni web sono limitate dalla capacità dei browser nel 
		caricare i dati e nel visualizzarli (loading e rendering). Infine non 
		essendo possibile distribuire queste applicazioni attraverso i più 
		popolari store, viene persa la visibilità e la possibilità di 
		monetizzazione, offerta dagli stessi. Per trarre profitto 
		dall'applicazione è quindi necessario trovare soluzioni alternative.
		
	\section{Applicazioni Ibride}
		Le applicazioni ibride combinano la convenienza dello sviluppo con HTML, 
		CSS e JavaScript con aspetti delle app native (da qui l'appellativo 
		``ibride''). Queste applicazioni vengono impacchettate in file 
		installabili proprio come le applicazioni native ma i linguaggi 
		utilizzati nella loro realizzazione sono ancora quelli impiegati nelle 
		applicazioni web. Le applicazioni ibride sono controllate da una web 
		view (UIWebView su iOS, WebView su Android e altri) in questo modo il 
		contenuto JavaScript e HTML è  presentato in un formato a schermo 
		intero usando il motore di rendering del browser nativo\footnote{Da 
		notare che non viene utilizzato il browser ma solo il motore.}. WebKit è 
		il motore di rendering del browser che è utilizzato su iOS, Android, 
		BlackBerry e altri. Quello che rende un'applicazione ibrida differente 
		da una web app è la possibilità di implementare un livello di astrazione 
		che espone le funzionalità del dispositivo (GPS, fotocamera, 
		accelerometro, filesystem, ecc...) attraverso API JavaScript.
		
		Esistono diversi framework che mettono a disposizione un buon insieme di 
		API per accedere alle funzionalità del dispositivo ma comunque c'è 
		ancora la necessità di far uso di software di terze parti per realizzare 
		l'interfaccia utente dell'applicazione. Proprio come le web app, le 
		applicazioni ibride fanno uso degli stessi framework per adempiere a 
		questo scopo (\hyperref[sec:webapp]{vedi~\ref{sec:webapp}}).
		
		Il poter utilizzare i linguaggi del web fa delle applicazioni ibride una 
		buona strategia nell'ottica dello sviluppo \crossplat tenendo in 
		considerazione anche il fatto che si ha la possibilità di accedere a 
		funzionalità specifiche del dispositivo. Con questo approccio la 
		distribuzione delle app può avvenire tramite i comuni store con 
		annessa possibilità di monetizzazione.
		
		Anche se lo stesso codice può essere utilizzato su diverse piattaforme, 
		è necessario programmare in ambienti di sviluppo opportunamente 
		configurati con i diversi SDK per realizzare i vari pacchetti. Ad oggi 
		esistono anche dei servizi online di compilazione cloud-based che 
		permetto di ovviare a questo inconveniente. Una difficoltà importante si 
		trova nella fase di debugging dove è necessario installare 
		l'applicazione sul dispositivo (al contrario delle web app che possono 
		essere testate direttamente nel browser) e ricorrere a software di terze 
		parti per poter poi avere solo a disposizione semplici strumenti per il 
		solito web debugging\footnote{Un software che permette di utilizzare 
		strumenti di web debugging per applicazioni ibride è WEINRE 
		(\url{http://people.apache.org/~pmuellr/weinre/docs/latest/Home.html}). 
		}. Da tener conto, quando si decide di sviluppare un'applicazione con 
		questo stile, è la possibilità che questa venga rifiutata nel momento 
		della pubblicazione sullo store; questo può accadere, per esempio, su 
		App Store dove esistono regole molto rigide che l'applicazione deve 
		rispettare riguardo a prestazioni, qualità e stile di programmazione.
		
	\section{Conclusioni}
		Ogni paradigma di sviluppo ha i suoi vantaggi e svantaggi; non esiste il 
		modo migliore per sviluppare un'applicazione ma piuttosto esiste una 
		scelta migliore tenendo conto di cosa si vuole ottenere e delle risorse 
		che si hanno a disposizione che possono riguardare sia la conoscenza dei
		linguaggi di programmazione sia il tempo e il denaro che si hanno a
		disposizione.
		
		Sviluppare un applicazione web può risultare più rapido e magari meno 
		costoso, ma questa soluzione è accettabile se non si desidera che il 
		prodotto finale venga distribuito via store e che si disponga di 
		altre soluzioni se si è interessati a trarre profitto dalla vendita.
		Questo \textit{modus operandi} è attuabile se si sviluppano applicazioni
		che non necessitino di utilizzare sensori del dispositivo mobile, ad
		esempio applicazioni di gestione di dati o di applicazioni connesse 
		come e-commerce. Se si è disposti ad accettare questi compromessi,
		un'applicazione web è sicuramente una buona soluzione dal punto di vista 
		dello sviluppo \crossplat stando attenti ai possibili problemi di 
		compatibilità tra i diversi browser che verranno impiegati dagli utenti.
		
		Se abbiamo la necessità, invece, di realizzare un'applicazione più
		complessa che sfrutti il dispositivo sottostante e non si vuol
		rinunciare alle, magari già familiari, tecnologie web allora una buona
		idea potrebbe essere sviluppare	un'applicazione ibrida. Questa
		soluzione, implicando la realizzazione di pacchetti installabili, offre
		la possibilità di vendere il prodotto finito sugli store traendone
		profitto. Per chi è esperto nelle tecnologie web e si vuole cimentare
		nella programmazione mobile, questa soluzione si rivela ottima prestando
		le dovute attenzioni, infatti possono sorgere problemi quando si accede
		alle funzionalità del dispositivo oppure alcune funzionalità desiderate
		potrebbero non essere implementabili\footnote{Un esempio sono i servizi
		che vengono eseguiti in background sui dispositivi Android che non sono
		implementabili in applicazioni ibride}.
		
		Se si deve realizzare un'applicazione molto responsiva come per esempio 
		un gioco con una pesante grafica oppure, se per scelte di mercato, si
		decide di supportare una sola piattaforma mobile, allora vale la pena
		intraprendere la strada di sviluppo in codice nativo. In questo modo non
		si incorre nel rischio che l'applicazione non venga accettata sullo
		store relativo e di sicuro si ottengono prestazioni più elevate rispetto
		alle altre due soluzioni. Inoltre se comunque si vogliono supportare più
		piattaforme, si può pensare di voltare lo sguardo su framework
		cross-compiling (\hyperref[sec:titanium]{vedi~\ref{sec:titanium}}) così
		da utilizzare il medesimo codice (non nativo, per esempio JavaScript)
		per creare applicazioni, a tutti gli effetti, native.

