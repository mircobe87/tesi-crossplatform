\chapter{Sviluppo Applicazioni Mobili}
	Vista la grande diffusione di dispositivi mobili, come smartphone e tablet, in grado di eseguire applicazioni scritte da terze parti, un'azienda potrebbe essere intenzionata a fornire i propri servizi anche attraverso queste applicazioni.
	
	Diversamente dalle applicazioni per desktop, dove per molto tempo la piattaforma di riferimento è stata Windows, il panorama delle piattaforme mobili è molto frammentato.
	I principali sistemi operativi mobili sono: iOS, Android e Windows Phone, seguiti dai meno diffusi Blackberry e Symbian, inoltre ne stanno nascendo continuamente di nuovi.
	
	Quando si decide di sviluppare un'applicazione mobile, è quindi necessario scegliere se sviluppare per una o più di queste piattaforme.
	Lo sviluppo su una sola piattaforma (sviluppo single-platform), è appropriato se si ha intenzione di sviluppare applicazioni che avranno un uso interno dove i dispositivi sono controllati o se per un strategia di mercato si è deciso di concentrarci sullo sviluppo per un unico produttore che adotta un'unica piattaforma.
	Anche se sviluppare per una singola piattaforma semplifica di molto il lavoro, questo pone dei rischi in quanto il successo dell'applicazione è legato al mantenimento del grado di preferenza degli utenti della piattaforma supportata.
	L'alternativa è quella di sviluppare l'applicazione supportando diverse piattaforme (sviluppo cross-platform), col vantaggio di raggiungere un bacino di utenza più ampio.
	
	Ci sono molti approcci differenti per lo sviluppo Cross-platform, ognuno con differenti vantaggi, ma la caratteristica che tutti hanno in comune è la possibilità di sviluppare la stessa (o molto simile) applicazione per diverse piattaforme mobili.
	 
	Principalmente è possibile sviluppare tre tipi di applicazioni, ognuno dei quali può essere usato sia per lo sviluppo single-platform, sia per quello cross-platform: Native, Applicazioni Web, Ibride, .
	Nei successivi paragrafi verranno spiegati nel dettaglio, e per ognuno saranno affrontati i pro e i contro, con particolare attenzione per lo sviluppo cross-platform.
	
	\section{Applicazioni Native}
		Lo sviluppo nativo di applicazioni si basa sull'utilizzo di SDK e di linguaggi di programmazione specifici per ogni piattaforma. Per esempio, le applicazioni native per iOS usano le API di Apple, Objective C e UIKit; su Android si utilizzano le API di Google, Java e la specifica sintassi XML per la UI; su Windows Phone si utilizzano le API di Microsoft, .NET e XAML.

Sviluppare applicazioni native permette di creare applicazioni molto performanti, tale caratteristica rende questa tecnica la più opportuna per lo sviluppo di giochi con una ricca grafica. Inoltre grazie alle API, fornite direttamente dal produttore, si ha un completo accesso al dispositivo, con la possibilità quindi di sfruttare i vari sensori (Fotocamera, GPS, Accelerometro). Infine è possibile distribuire l'applicazione sfruttando i vari store delle piattaforme, questo permette di avere maggiore pubblicità per l'applicazione grazie all'enorme ecosistema che si è venuto a creare intorno ad essi.

Di contro questa tecnica non è molto adatta allo sviluppo di applicazioni cross-platform. Sviluppare la stessa applicazione per diverse piattaforme richiede che diversi team di sviluppatori scrivano la stessa applicazione più volte in linguaggi diversi e in ambienti di sviluppo differenti.
Il problema inoltre cresce con il numero di piattaforme che si intende supportare. Un azienda, per ogni piattaforma, deve spendere lo stesso tempo, sforzo e denaro per produrre e testare l'applicazione. E lo stesso vale anche per il mantenimento della app.
Un' altro importante fattore in ogni progetto di sviluppo di applicazioni native cross-plaftorm è la disponibilità di programmatori con le giuste conoscenze. Lo sviluppo di applicazioni native richiede l'assunzione di sviluppatori che conoscano molti linguaggi diversi, oppure l'assunzione di molti sviluppatori con conoscenze opportune per ogni piattaforma.
 
Esistono strumenti di cross-compiling che possono essere una soluzione parziale del problema consentendo di scrivere l'applicazione in un solo linguaggio (per esempio javaScript) e poi compilarla in applicazione nativa. Questa soluzione aiuta a ridurre il numero di differenti linguaggi di programmazione che gli sviluppatori devo saper maneggiare ma comunque non elimina la necessità di dover compilare differenti applicazioni per ogni piattaforma. Cross-compiling inoltre introduce un nuovo livello di astrazione tra il codice della app e il runtime; di questo bisogna tenere conto quando si vanno a fare operazioni di ottimizzazione, testing e di debugging.

	\section{Applicazioni Web}
		Le Applicazioni Web (chiamate anche Browser App o Web App) non sono altro che siti web ottimizzati per i dispositivi mobili. Il loro uso è reso possibile dai moderni browser mobili come: Mobile Safari (sui dispositivi iOS), Google Chrome per Android (sui dispositivi Android 4+) e Internet Explorer 10 (sui dispositivi Windows Phone 8).
		
		Per realizzare questo tipo di applicazione viene sfruttata la potenza di HTML5, CSS3 e JavaScript. HTML5 è la quinta versione del linguaggio HTML, il quale fornisce gli elementi di base che costituiscono le pagine web. HTML5 supporta i contenuti multimediali, ed è stato creato con l'obiettivo di poter essere eseguito da qualsiasi browser mobile.
		Una delle tante novità di HTML5 è la cosiddetta "application cache": attraverso questo strumento è possibile dire al browser quali file salvare (non nella cache "normale" ma in un cache apposita, definita per l'appunto "application cache").
		Il funzionamento dell'application cache è piuttosto semplice: sostanzialmente, si crea un file all'interno del quale si specifica la lista di file che il browser deve salvare nella sua "memoria" e che deve mostrare anche quando si è offline.
		Di conseguenza il vantaggio principale della "application cache", rispetto a quella tradizionale, è il controllo: mentre nella cache "normale" è il browser a decidere (più o meno) quali sono i file da salvare e tenere in cache, con lo strumento offerto da HTML5 possiamo dire noi con precisione quali risorse tenere in memoria, possiamo infatti decidere di memorizzare pagine sulle quali non si è ancora navigato, fino addirittura a memorizzare interi script. In questo modo le applicazioni web sono in grado di essere eseguite anche quando il dispositivo mobile è offline.

Scrivere un'applicazione web di successo, non è solo una questione di rispettare le specifiche, ma dipende anche dall'implementazione del browser del dispositivo. E' da vedere quanto sia nell'interesse di case come Apple e Google tenere i loro browser aggiornati continuamente con gli standard, infatti ci sarebbe poi il rischio che HTML5 diventi uno standard per le applicazioni e che quindi ci rimettano gli store\citep{White:Native-vs-Html}. 

Sfruttando opportunamente CSS3, HTML5 e JavaScript, gli sviluppatori possono ottimizzare la loro applicazione presentandola con un aspetto diverso a seconda del sistema operativo mobile che intendono supportare, in modo da rendere l'esperienza utente il più possibile simile a quella attesa, oppure possono progettare un'unica interfaccia grafica che viene utilizzata su ogni piattaforma. A tale scopo sono nati numerosi framework.%riferimento al paragrafo sui framework per la UI

Questa tecnica è la più veloce per realizzare applicazioni mobili cross-platform. Essendo eseguite direttamente nel browser non c'è la necessità di creare il pacchetto nativo da installare su ogni piattaforma, liberando così gli sviluppatori dal vincolo di lavorare in differenti ambienti di sviluppo. Le applicazioni web hanno il vantaggio di poter fruire di aggiornamenti direttamente attraverso il browser, è infatti sufficiente che lo sviluppatore aggiorni il codice della pagina web online. Non sono quindi necessari passaggi extra per la distribuzione dell'aggiornamento.
Rispetto alle applicazioni native c'è il vantaggio della maggior disponibilità di programmatori con conoscenze dei linguaggi CSS3, HTML5 e JavaScript, rispetto a quelli con conoscenze specifiche per i linguaggi Objective-C, Java e .NET.

Il più grande svantaggio di questo genere di applicazioni è che non possono accedere ad una vasta gamma di sensori e API dei vari dispositivi.
La promessa di HTML5 di scrivere una volta ed eseguire ovunque non è totalmente rispettata, infatti, essendoci molti browser diversi e molte versioni degli stessi browser (chrome è stato aggiornato 21 volte in due anni, con conseguente modifica dell'interpretazione di web app)\citep{White:Native-vs-Html} i vari dispositivi accedono in modo diverso alle applicazioni web.
La conseguenza è che gli sviluppatori devono creare versioni diverse per browser diversi, con la conseguente perdita di parte dei benefici argomentati per lo sviluppo di web app.
Un altro grande svantaggio si ha nelle performance che sono sicuramente inferiori a quelle delle applicazioni native, dato che le prestazioni delle applicazioni web sono limitate dalla capacità dei browser nel caricare i dati e nel visualizzarli (loading e rendering).
Infine non essendo possibile distribuire queste applicazioni attraverso i più popolari app store, viene persa la visibilità e la possibilità di monetizzazione, offerta dagli stessi.
Per trarre profitto dall'applicazione è quindi necessario trovare soluzioni alternative.
		
	\section{Applicazioni Ibride}
		Le applicazioni ibride combinano la convenienza dello sviluppo con HTML, CSS e JavaScript con la potenza delle app native. Con questa tecnica le applicazioni sono sviluppate con la stessa tecnologia usata per le applicazioni Web ma l'applicazione risultante è impacchettata in una “shell” che estende la potenza della web app. La shell nativa agisce come un proxy che consente a JavaScript di accedere ad un vasto insieme di API e di sensori del dispositivo cose che normalmente non è possibile tramite un browser e in più, tale shell, permette alla app di essere distribuita attraverso i normali app store.
Apache Cordova, conosciuta anche come PhoneGap (nome della versione “brandizzata” della Adobe) è il più popolare contenitore open source di app ibride. Lavorando con questo contenitore, gli sviluppatori possono ottenere la produttività di una app Browser mantenendo la possibilità di accedere a molte API e sensori del dispositivo come fotocamera, bussola, accelerometro e tanti altri.
Diversamente dalle Browser App che vengono semplicemente distribuite attraverso il web browser, le app ibride devono essere compilate per ogni singola piattaforma e installate sul dispositivo. Il codice dell'applicazione può essere lo stesso per ogni piattaforma ma sono necessarie separate  configurazioni di compilazione per poter supportare tutte le varie piattaforme. Nuovi strumenti cloud-based come Icenium e PhoneGap Build aiutano ad aggirare questo problema gestendo il processo di compilazione e distribuendo l'applicazione ibrida per le varie piattaforme.
Un applicazione ben fatta e realizzata con i giusti strumenti può risultare agli utenti indistinguibile dalle applicazioni native.
		
	\section{Conclusioni}
	NOTA: Browser App e App Ibride girano “nativamente” sui dispositivi. Anche se il termine “app nativa” si riferisce allo SDK nativo, è importante notare che le Browser App e le App Ibride girano su runtimes mobili nativi e non su plug-in.
		Visti i pro e contro derivanti dai diversi approcci, considerazioni su
		quale di questi metodi scegliere a seconda dei proprio obiettivi.
		
	
		
	