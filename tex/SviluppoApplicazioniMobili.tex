\chapter{Sviluppo Applicazioni Mobili}
	Vista la grande diffusione di dispositivi mobili, come smartphone e tablet, in grado di eseguire applicazioni scritte da terze parti, un'azienda potrebbe essere intenzionata a fornire i propri servizi anche attraverso queste applicazioni.
	
	Diversamente dalle applicazioni per desktop, dove per molto tempo la piattaforma di riferimento è stata Windows, il panorama delle piattaforme mobili è molto frammentato.
	I principali sistemi operativi mobili sono: iOS, Android e Windows Phone, seguiti dai meno diffusi Blackberry e Symbyan, inoltre ne stanno nascendo anche di nuovi.
	
	Quando si decide di sviluppare un'applicazione mobile, è quindi necessario scegliere se sviluppare per una o più di queste piattaforme.
	Lo sviluppo su una sola piattaforma (single-platorm), è appropriato se si ha intenzione di sviluppare applicazioni che avranno un uso interno dove i dispositivi sono controllati o se per un strategia di mercato si è deciso di concentrarci sullo sviluppo per un unico produttore che adotta un unica piattaforma.
	Anche se sviluppare per una singola piattaforma semplifica di molto il lavoro, questo pone dei rischi in quanto il successo dell'applicazione è legato al mantenimento del grado di preferenza degli utenti della piattaforma supportata.
	L'alternativa è quella di sviluppare l'applicazione supportando diverse piattaforme (Cross platform), col vantaggio di raggiungere un bacino di utenza più ampio.
	
	Ci sono molti approcci differenti per lo sviluppo cross-platform, ognuno con differenti vantaggi ma la caratteristica che tutti hanno in comune è la possibilità di sviluppare la stessa (o molto simile) applicazione per diverse piattaforme mobili.
	 
	Cosa prendere in considerazione nel momento in cui si decide di sviluppare
	una app; dalle piattaforme da supportare ai tempi e costi che si intendono
	affrontare.
	
	\section{Applicazioni Native}
		Cosa è una app nativa. Cosa comporta lo sviluppo di app native
		nell'ottica di supportare diverse piattaforme. Presentazione
		dell'approccio "Titanium" che permette la realizzazione di app nativa a
		partire da codice comune javaScript.
		
	\section{Applicazioni Ibride}
		Cosa si intente per app ibrida. Cosa comporta lo sviluppo di questo
		genere di app. Pro e contro.
		
	\section{Applicazioni Web}
		Cosa sono le applicazioni web. Pro e contro.
		
	\section{Conclusioni}
		Visti i pro e contro derivanti dai diversi approcci, considerazioni su
		quale di questi metodi scegliere a seconda dei proprio obiettivi.
		
	
		
	