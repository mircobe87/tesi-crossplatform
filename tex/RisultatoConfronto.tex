\chapter{Risultato del confronto}
    Descriviamo ora il risultato del confronto tra i due framework valutando
    le seguenti caratteristiche: interfaccia grafica, prestazioni,
    funzionalità, strumenti per lo sviluppo, curva di apprendimento,
    \crossplat{}. Ricordiamo ancora una
    volta che l'applicazione è stata testata soltanto su dispositivi Android.
        \subsubsection{Interfaccia grafica}
        La qualità dell'interfaccia grafica è sicuramente migliore in \tisdk{},
        soprattutto per il fatto che è possibile accedere agli elementi
        grafici nativi delle varie piattaforme. \kendomob{} prova tramite i
        suoi file \css{} ad emulare l'aspetto nativo, ma effettivamente si
        notano ancora alcune differenze; dalle dimostrazioni disponibili sulla
        documentazione di questo framework, invece, sembra che per iOS l'aspetto
        rispecchi in maniera più fedele quello nativo. Integrare \pg{} con
        altri framework per la realizzazione dell'interfaccia grafica potrebbe
        comunque risolvere in parte il problema dell'aspetto.
        \subsubsection{Prestazioni}
        L'applicazione ibrida fornisce buone prestazioni in quasi tutte
        le sue funzionalità, soltanto la mappa mostra problemi di fluidità
        nelle operazioni di zoom o di spostamento, va però detto che queste
        prestazioni sono anche legate al servizio usato per la mappa.
        \tisdk{} ha permesso la realizzazione di un'applicazione molto fluida
        anche nell'uso della mappa. Per cui anche in questo campo ci sentiamo
        di consigliare \tisdk{}.
        \subsubsection{Funzionalità}
        Entrambi i framework sono risultati validi nel realizzare le
        funzionalità che ci erano state richieste. Di base \tisdk{} fornisce
        sicuramente più strumenti, ma l'estendibilità di \pg{} tramite plugin
        permette di ridurre questo gap, molti di questi plugin sono
        disponibili in rete e avendo conoscenze del linguaggio nativo è
        possibile crearsene di propri. L'unica vera differenza
        trovata è stata nell'impossibilità di implementare il servizio di
        background (vedi \ref{subsec:nuova}) con \pg{}, questo problema
        riguarda proprio il concetto con
        il quale sono costruite queste applicazioni: è si possibile lanciare
        in \js{}
        un servizio in background che possa essere eseguito anche quando
        l'applicazione è stata chiusa (in rete è disponibile un plugin proprio
        per questo scopo) ma, il codice del servizio deve essere comunque
        scritto in linguaggio nativo in quanto, un servizio scritto in codice
        \js{}, essendo interpretato all'interno della Web View
        dell'applicazione, non potrebbe essere eseguito quando questa viene
        chiusa.
        \tisdk{} invece convertendo il codice \js{} in codice nativo non
        soffre di questo problema.
        \subsubsection{Strumenti per lo sviluppo}
        Di per se \pg{} offre soltanto la CLI per creare, compilare,
        installare l'applicazione sul dispositivo e aggiungervi plugin. Non
        offre quindi un proprio IDE e nemmeno particolari strumenti per il
        debug. Se si decide di non usare il servizio \pg{} Build avendo dovuto
        installare gli SDK nativi per le piattaforme di destinazione è
        possibile sfruttare gli IDE nativi e gli strumenti che essi offrono,
        come il logcat di Android. Inoltre esistono altre soluzioni create
        appositamente per fornire strumenti di sviluppo aggiuntivi a quelli di
        \pg{}, è il caso di Icenium\footnote{al momento della scrittura di
        questa tesi è stato rinominato in app builder.
        Maggiori informazioni possono essere trovate sul sito
        \url{http://www.telerik.com/appbuilder}} che in particolare offre un
        ICE (Integrated Cloud Environment), cioè un IDE che però lavora su una
        cloud. Per essere più chiari, Icenium fornisce un moderno ambiente di
        code-editing, con un integrata analisi del codice in tempo reale,
        debugging, sintassi colorata e formattata, completamento di istruzioni,
        refactoring e navigazione nel codice. Con Icenium la compilazione
        viene fatta in remoto come con \pg{} Build, e inoltre l'applicazione
        può essere testata sia su Android che su iOS senza possedere una
        licenza di sviluppatore (che ovviamente serve comunque per distribuire
        l'applicazione attraverso gli store), il debug viene fatto sempre
        attraverso gli strumenti del web, ma il tutto è integrato in Icenium.
        Questo potente ICE può essere usato direttamente da un qualsiasi browser
        (con alcune limitazioni alle funzionalità) o tramite un apposito
        programma per Windows. Per finire Icenium integra di default il
        framework \kendomob{} (ma può essere usato anche con altri). Non
        l'abbiamo utilizzato per lo sviluppo dell'applicazione ibrida soltanto
        perché volevamo capire bene cosa offrisse da solo \pg{} e perché il
        suo utilizzo è gratuito solo per il primo mese. Un'altra soluzione di
        questo tipo è offerta dalla Intel con IntelXDK disponibile sul sito
        \url{http://xdk-software.intel.com/}. Siamo inoltre sicuri che,
        anche se abbiamo trovato il servizio \pg{} Build ancora acerbo, nel
        futuro prossimo sarà un ottimo strumento per evitare la complessa
        configurazione in locale di tutti gli SDK. In ultimo citiamo la
        documentazione di \pg{} che è ben fatta ma a causa del fatto che venga
        copiata da quella di Cordova e poi riadattata nel tempo a \pg{}
        presenta lacune ed è molto confusionaria sulla descrizione dell'uso
        del framework.

        \tisdk{} come detto offre un proprio IDE che permette anche di fare
        debugging del codice \js{} con la possibilità di interrompere
        l'esecuzione dell'applicazione e di eseguire la sua logica passo passo
        in modo da individuare più facilmente gli errori. Ad ogni modo anche
        con \tisdk{} il maggior strumento di testing è stato il logcat di
        Android che però è ben integrando nell'IDE. Abbiamo trovato
        questi strumenti abbastanza utili, anche se sia in ambiente linux che
        windows abbiamo avuto non pochi problemi di configurazione.
        \subsubsection{Curva di apprendimento}
        Sicuramente il tempo per iniziare a scrivere applicazioni ibride con
        \pg{} è molto breve, molto dipende anche dal framework usato per
        l'interfaccia grafica, ma \kendomob{} si è rivelato tra i più semplici.
        L'uso delle API è molto semplice ed intuitivo, per non parlare del
        fatto che se si desidera creare applicazioni \crossplat{} la curva di
        apprendimento è praticamente indipendente dal numero di piattaforme
        che si intende supportare.

        \tisdk{} è sicuramente più complicato da apprendere e soprattutto
        bisogna familiarizzare con alcuni concetti tipici delle piattaforme
        native, con la conseguenza che spesso ci si ritrova a leggere la
        documentazione nativa della piattaforma di destinazione. Per
        questo motivo, in un ottica di sviluppo \crossplat{}, sarà necessario
        spendere più tempo per l'apprendimento a seconda del numero di diverse
        piattaforme per le quali si è deciso di sviluppare applicazioni. Ad
        ogni modo \tisdk{} è accompagnato da una ricca documentazione ben
        fatta e da una serie di guide che descrivono concetti
        importanti relativi all'uso di questo framework; oltre a questo
        \tisdk{} fornisce un insieme di API ben strutturato che, una volta
        presa familiarità con questo framework, consente di trovare subito
        quello di cui si ha bisogno.
        \subsubsection{Cross-platform}
        L'obbiettivo principale del tirocinio era proprio quello di valutare
        come i due framework si comportassero nello sviluppo di applicazioni
        \crossplat{}, purtroppo non avendo la possibilità di testare
        effettivamente l'applicazione su altre piattaforme le conclusioni che
        possiamo fornire si basano soltanto su ciò che abbiamo appreso dalle
        documentazioni. L'applicazione ibrida permette sicuramente di
        condividere la maggior parte del codice sia per quanto riguarda la
        grafica che la logica, questo grazie alla consistenza delle API \pg{}
        e alla adattività di quelle \kendomob{}.

        Con \tisdk{}, invece, la quantità di codice che è possibile
        condividere tra le varie piattaforme è minore soprattutto se intendiamo
        utilizzare delle funzionalità particolari come abbiamo fatto per
        la action bar di Android. Ad ogni modo la differenziazione del codice
        è resa semplice dal paradigma MVC e dalle API fornite dal framework
        allo scopo.
