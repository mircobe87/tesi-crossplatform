\documentclass[12pt, a4paper, twoside, openright, titlepage]{book}

% soliti pacchetti per la lingua italia 
\usepackage[utf8]{inputenc}
\usepackage[italian]{babel}

% pacchetto potente per il frontespizio
\usepackage[write]{frontespizio}

% per settare l'interlinea a una line a e mezzo tramite il comando \onehalfspacing
\usepackage{setspace}

% definisco l'ambiente "abstract" per la classe "book" che si sta usando.
\usepackage{fancyhdr}
\newcommand{\fncyblank}{\fancyhf{}}
\newenvironment{abstract}{\cleardoublepage\fncyblank\null\vfill\begin{center}\bfseries\abstractname\end{center}}{\vfill\null}

% pacchetto per la personalizzazione della bibliografia
\usepackage[numbers]{natbib}
\usepackage{url}

% pacchetto per link ipertestuali cliccabili in documento pdf, seguito dalle impostazioni
\usepackage{hyperref}
\hypersetup{
    pdftitle={Sviluppo Crossplatform Di Applicazioni Mobili},    % title
    pdfauthor={Mirco Bertelli, Maurizio Pratesi},     % author
    pdfsubject={Tesi Di Laurea Triennale},   % subject of the document
    colorlinks=true,        % false: boxed links; true: colored links
    linkcolor=black,          % color of internal links (change box color with linkbordercolor)
    citecolor=blue,        % color of links to bibliography
    filecolor=blue,      % color of file links
    urlcolor=blue          % color of external links
}

\begin{document}
\onehalfspacing % imposta l'interlinea ad una riga e mezzo

	\begin{frontespizio}
	\Universita{Pisa}
	%\Logo[5cm]{unipi}
	\Filigrana[height=6cm,before=11,after=53]{../graphics/unipi}
%	\Facolta{Scienze, Matematiche, Fisiche e Naturali}
	\Dipartimento{Informatica}
	\Corso[Laurea]{Informatica}
	\Annoaccademico{2012--2013}
	%\Titoletto{Tesi di laurea triennale}
	\Titolo{Sviluppo Cross platform Di Applicazioni Mobili}
	\Sottotitolo{Analisi su differenti approcci nello sviluppo crossplatform di applicazioni mobili}
	\Candidato{Mirco Bertelli}
	\Candidato{Maurizio Pratesi}
	\Relatore{Paolo Milazzo}
	\NRelatore{Tutore Accademico}{Tutori Accademici}
	\Correlatore{Simone Gianfranceschi}
	\NCorrelatore{Tutore Aziendale}{Tutori Aziandali}
	\Margini{2.5cm}{1.5cm}{2.5cm}{1.5cm}
\end{frontespizio}
\newpage
\mbox{}
 % includo il frontespizio
	\begin{flushright}
	\null\vspace{\stretch{4}}
	
		La dedica va qui.\\
		Questa è la seconda riga.
	
	\vspace{\stretch{1}}\null
\end{flushright}
\newpage
       % includo la dedica
	\begin{abstract}

    Negli ultimi anni sono stati sviluppati e resi disponibili una serie di
    framework, che consentono di sviluppare applicazioni per varie piattaforme
    mobili senza dover necessariamente scrivere codice nativo. Con la seguente
    tesi si vogliono documentare e mostrare i risultati dell'analisi comparativa
    effettuata su questi framework, con lo scopo di fare chiarezza sui pro e i
    contro di queste soluzioni in vari contesti applicativi. Questo studio è
    stato intrapreso come oggetto di tirocinio presso l'azienda informatica
    Intecs S.p.A di Pisa.

\end{abstract}
\newpage
     % include il sommario
	
	\tableofcontents       % ecco l'indice
	
	\chapter{Titolo del Capitolo Uno}
	Questo è il contenuto del primo capitolo.
	\section{Titolo di Una sezione}
		Questa è una sezione del primo capitolo.
		\subsection{Titolo della prima sottosezione}
			Questa è la sottosezine della prima sezione del capitolo uno.
		\subsection{Titolo della seconda sottosezione}
			Questa è la seconda sottosezine della prima sezione del capitolo uno.
	\section{Titolo di un'altra sezione}
		Questa è la seconda sezione del capitolo uno.

\chapter{Titolo del Capitolo Due}
	Questo è il contenuto del secondo capitolo.
	Prova di riferimento bibliografico al sito di Phonegap \citep{Web:PG}.
	Prova di riferimento bibliografico alla doc online di Phonegap \citep{Doc:PG}.


\chapter{Titolo del Capitolo Tre}
	Questo è il contenuto del terzo capitolo.


\chapter{Titolo del Capitolo Quattro}
	Questo è il contenuto del quarto capitolo.
    % includo uno scheletro di anteprima di tutta la relazione
	                       % ma poi questo verrà sostituito con l'inclusione di tutti
	                       % i capitoli.
	
	% un po' di capitoli...
	
	\bibliography{../bib/biblio}
	\bibliographystyle{plainnat}
	
\end{document}
