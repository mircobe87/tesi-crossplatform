\chapter{L'Applicazione}
	Per avere un confronto più significativo il tutore aziendale ci ha proposto 
	un'applicazione che dimostri l'interazione con il dispositivo e che quindi 
	usi le varie API dei framework.
	
	Per avere un confronto più significativo, il tutore aziendale ci ha proposto
	la realizzazione di un'applicazione che interagisse, in maniera sostanziale,
	con le funzionalità offerte dal dispositivo in modo da poter valutare meglio
	la potenzialità delle API offerte dai due framework in analisi.
	
	L'applicazione che ci è stata commissionata permette all'utente di segnalare 
	problemi di degrado ambientale con immagini e testo georeferenziati.
	In particolare l'applicazione inizialmente doveva permette all'utente di:
	\begin{itemize}
		\item registrarsi nel sistema al primo avvio e di autenticarsi in quelli
		      successivi;
		\item visualizzare su una mappa le proprie segnalazioni e quelle inviate
		      degli altri utenti;
		\item comporre e inviare su un server una nuova segnalazione
		      georeferenziata inserendo una descrizione testuale e una foto del
		      degrado scattata con il dispositivo;
		\item visualizzare una lista riassuntiva delle proprie segnalazioni
		      inviate;
		\item sceglere un segnalazione e visualizzarne i dettagli.
	\end{itemize}
	Durante la fase di sviluppo però, in occasine della revisione periodica del
	nostro lavoro svolto, sono state richieste nuove funzionalità aggiuntive; in
	particolare ci è stato chiesto di aggiungere:
	\begin{itemize}
		\item una lista generale che mostrasse le ultime segnalazioni inviate da
		      qualsiasi utente;
		\item un meccanismo di salvataggio locale delle proprie segnalazioni e
		      di tutte quelle visualizzate in dettaglio permettendo così
		      all'applicazione un parziale funzionamento anche in assenza di
		      connessione Internet;
		\item la possibilità di ritrovare sulla mappa una segnalazione partendo
		      dalla visualizzazione dei suoi dettagli.
	\end{itemize}
	
	La registrazione e l'autenticazione dell'utente sul server doveva avvenire 
	utilizzando come credenziali di accesso il codice IMEI o il numero di
	telefono, che dovevano essere letti automaticamente dall'applicazione,
	senza necessità d'interazione con l'utente. Abbiamo però deciso, inoltre, di
	dare all'utente, che si stà registrando, la possibilità di sceglersi un
	``nickname'' univoco\footnote{Il controllo dell'univocità è stato demandato
	al lato server e verrà descritto più in dettaglio nella sezione appostita.}
	e di inserire il proprio indirizzo e-mail con l'idea di aggiungere
	succesivamente queste informazioni in tutte le sue segnalazioni; in questo
	modo un altro utente che visualizza una certa segnalazione può vedere chi
	l'ha realizzata e, se vuole, sarà in grado di conttattarlo tramime posta
	elettronica. Per semplicità abbiamo deciso che una volta scelto il nickname
	questo sarà permantente, diversamente abbiamo aggiunto nell'applicazione una
	schermata tramite la quale sarà possibile aggiornare il proprio indirizzo
	e-mail.
	
	L'applicazione, come ovvio, sarà composta di una parte client e di una parte
	server. Il lato client è la parte dove si concentrano i nostri studi e dove
	mostreremo come sono state implementate le funzionalità richieste
	utilizzando i due framework presi in esame. L'analisi del lato server non
	è scopo di questa attività di tirocinio e quindi ne daremo solo una
	semplice descrizzione indicando il modo di funzionamento e le tecnologie
	impiegate, nonché il modo in cui è stato realizzato.
	
	\section{Lato Server}
		Lo scopo principale del lato server di questa applicazione è quello di
		gestire un data-base che andrà a contenere tutte le segnalazioni inviate
		dagli utenti che vi interagiranno attraverso il lato client. Il
		data-base dovrà essere in grado di rispondere a query riguardanti
		certamente le coordinate geografiche, ma anche i nickname e la data e
		ora delle varie segnalazioni. Un secondo compito importante che dovrà
		svolgere sarà quello di gestire le identità dei diversi utenti e di
		controllare l'univocità del nickname utilizzato in fase di
		registrazione.
		
		Una caratteristica in più che non era richiesta ma che ci sembrava
		opportuno avere era l'indipendenza della realizzazione del server
		rispoetto ai due framework utilizzati nell'implementare il lato client.
		In questo modo entrambe le implementazioni del lato client avrebbero
		dialogato con la stessa implementazione del lato server.
		
		Per la realizzazione di tutto questo, dopo una fase di ricerca, abbiamo
		mirato a due prodotti open source vista la loro semplicità d'uso:
		Apache CouchDB\texttrademark{} e Node.js.
		\begin{description}
			\item[Apache CouchDB\texttrademark{}] è un data-base non relazionale
				implementato sul paradigma Map-Reduce. La comunicazione con esso
				avviene mediante opportuni messaggi HTTP rendondo semplice la
				richista di risoluzione di query. Sempre attraverso HTTP è
				possibile inviare date e ricevere dati strutturandoli in
				formato JSON; inoltre è possibile allegare a documenti JSON
				file multimendiali di ogni genere. Questo data-base fornisce
				anche un interfaccia utente Web così da semplificare molto la
				sua configurazione iniziale. Apache CouchDB\texttrademark{} non
				consente nativamente di eseguire query spaziali ma grazie
				all'estenzione Geocouch anche questa funzionalità è stata resa
				disponibile.
			\item[Node.js] è una piattaforma realizzata sul motore \js{} V8 di
				Google Chrome per la realizzazione facile e veloce di
				applicazioni di rete. La potenza di questa piattaforma è anche
				nel fatto che è supportata da una enorme quantità di moduli
				che permettono di realizzare con poco codice applicazioni piuttosto
				complesse. In più, come Apache CouchDB\texttrademark{}, tutto 
				questo è open source.
		\end{description}
		
		
	
	\section{Lato Client}
		descrizione funzionalità e struttura applicazione. per ogni funzionalità
		si mostra il confronto tra i due framework. In questa sezione, se 
		necessario, si inseriranno qualche screenshot delle 2 app.
