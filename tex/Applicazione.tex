\chapter{L'Applicazione}
	Per avere un confronto più significativo il tutore aziendale ci ha proposto 
	un'applicazione che dimostri l'interazione con il dispositivo e che quindi 
	usi le varie API dei framework.
	
	L'applicazione che ci è stata commissionata permette all'utente di segnalare 
	problemi di degrado ambientale con immagini e testo georeferenziati.
	In particolare l'applicazione inizialmente doveva permette all'utente di:
	\begin{itemize}
	\item Registrarsi nel sistema al primo avvio e di autenticarsi in quelli successivi
	\item Visualizzare su una mappa le proprie segnalazioni e quelle degli altri utenti
	\item Comporre e inviare su un server una nuova segnalazione georeferenziata inserendo 
	una descrizione testuale e una foto del degrado scattata con il dispositivo
	\item Visualizzare una lista riassuntiva dalla quale poter accedere alle proprie segnalazioni
	\end{itemize}
	Durante lo sviluppo dell'applicazione ci sono state richieste di aggiungere 
	nuove funzionalità:
	\begin{itemize}
	\item Una lista che mostrasse le ultime segnalazioni fatte da qualsiasi utente
	\item Salvataggio in locale delle proprie segnalazioni e di quelle 
	visualizzate, permettendo così all'applicazione un parziale funzionamento anche in 
	assenza di connessione di rete
	\item Ritrovare sulla mappa una segnalazione partendo dalla sua visualizzazione
	\end{itemize}
	
	La registrazione e l'autenticazione dell'utente sul server doveva avvenire 
	utilizzando come credenziali di accesso il codice IMEI o il numero di telefono, 
	che dovevano essere letti automaticamente dall'applicazione, senza necessità 
	di interazione con l'utente.
	Abbiamo però modificato questo aspetto aggiungendo in fase di registrazione 
	la possibilità di scegliere un nome utente univoco 
	(che non potrà più essere cambiato) e un indirizzo mail (che potrà essere 
	cambiato in qualsiasi momento). Le credenziali per l'accesso restano il codice 
	IMEI o il numero di telefono, ma nome utente e mail compariranno automaticamente 
	nelle segnalazioni in modo da poter contattare eventualmente un segnalatore per 
	chiedere approfondimenti sul degrado segnalato.
	
	L'applicazione dunque si divide in una parte client e in una parte server.
	
	\section{Lato Server}
		descrizione funzionalità del server e dei software utilizzati per la sua
		realizzazione.
	
	\section{Lato Client}
		descrizione funzionalità e struttura applicazione. per ogni funzionalità
		si mostra il confronto tra i due framework. In questa sezione, se 
		necessario, si inseriranno qualche screenshot delle 2 app.
